\documentclass{article}
\usepackage{latexsym}
\usepackage[utf8]{inputenx}
\usepackage[spanish]{babel}
\usepackage{graphicx}
\usepackage{anysize}
\usepackage{amsmath}
\usepackage{amssymb}
\usepackage{float}
\setlength{\skip\footins}{5cm}
\usepackage{lscape}
\usepackage{verbatim}
\usepackage{moreverb}
\usepackage{url}
\usepackage{enumitem}
\usepackage{multicol}
\let\verbatiminput=\verbatimtabinput
\usepackage[nottoc,numbib]{tocbibind}
\setcounter{tocdepth}{4}
\setcounter{secnumdepth}{4}
\usepackage{color}
\definecolor{gray97}{gray}{.97}
\definecolor{gray75}{gray}{.75}
\definecolor{gray45}{gray}{.45}
\usepackage{array}
\usepackage[table]{xcolor}


\usepackage{listings}

\lstset{ frame=Ltb,
framerule=0pt,
aboveskip=0.5cm,
framextopmargin=3pt,
framexbottommargin=3pt,
framexleftmargin=0.4cm,
framesep=0pt,
rulesep=.4pt,
backgroundcolor=\color{gray97},
rulesepcolor=\color{black},
%
stringstyle=\ttfamily,
showstringspaces = false,
basicstyle=\small\ttfamily,
commentstyle=\color{gray45},
keywordstyle=\bfseries,
%
numbers=left,
numbersep=15pt,
numberstyle=\tiny,
numberfirstline = false,
breaklines=true,
}



\lstdefinestyle{consola}
{basicstyle=\scriptsize\bf\ttfamily,
backgroundcolor=\color{gray75},
}

\lstdefinestyle{C}
{language=C,
}


\marginsize{2cm}{2cm}{.5cm}{3cm} 

\begin{document}

 
\begin{titlepage}

\newcommand{\HRule}{\rule{\linewidth}{0.5mm}} % Defines a new command for the horizontal lines, change thickness here

\center % Center everything on the page
 
%----------------------------------------------------------------------------------------
%	HEADING SECTIONS
%----------------------------------------------------------------------------------------

\textsc{\LARGE Universidad De Buenos Aires}\\[1.5cm] % Name of your university/college
\textsc{\Large Facultad De Ingeniería}\\[0.5cm] % Major heading such as course name
\textsc{\large 66.20 Organización De Computadoras}\\[0.5cm] % Minor heading such as course title

%----------------------------------------------------------------------------------------
%	TITLE SECTION
%----------------------------------------------------------------------------------------

\HRule \\[0.4cm]
{ \huge \bfseries Trabajo Práctico 1}\\[0.4cm] % Title of your document
\HRule \\[1.5cm]
 
%----------------------------------------------------------------------------------------
%	AUTHOR SECTION
%----------------------------------------------------------------------------------------

% If you don't want a supervisor, uncomment the two lines below and remove the section above
\Large \emph{Integrantes:}\\
Daniel \textsc{Fernandez} - 93083\\ % Your name
Nicolas \textsc{Ortoleva} - 93196\\ % Your name
Maximiliano \textsc{Schultheis} - 93285\\[5cm] % Your name

%----------------------------------------------------------------------------------------
%	LOGO SECTION
%----------------------------------------------------------------------------------------

\includegraphics[scale=0.5]{img/UBA.jpg}\\[1cm] % Include a department/university logo - this will require the graphicx package

%----------------------------------------------------------------------------------------
%	DATE SECTION
%----------------------------------------------------------------------------------------

{\large \text \em {6 de Mayo de 2014}}\\[3cm] % Date, change the \today to a set date if you want to be precise
 
%----------------------------------------------------------------------------------------

\vfill % Fill the rest of the page with whitespace

\end{titlepage}

\tableofcontents

\newpage

\section{Diseño e implementación}





\section{Comandos de compilación}
\subsection{Makefile}
\lstinputlisting[language=sh]{../src/makefile para entregar/makefile}

\section{Pruebas realizadas}
\subsection{Primeras pruebas}
\lstinputlisting[language=sh]{../src/pruebas_cortas.sh}
\subsection{Prueba de archivos aleatorios}
\lstinputlisting[language=sh]{../src/prueba.sh}

\section{C\'odigo Fuente}
\subsection{C\'odigo fuente C tp1.c}
\lstinputlisting[language=C]{../src/tp1.c}
\subsection{C\'odigo de b16.h}
\lstinputlisting{../src/b16.h}
\subsection{C\'odigo assembly MIPS b16.S}
\lstinputlisting{../src/b16.S}
\subsection{C\'odigo assembly MIPS b16.S}
\lstinputlisting{../src/b16.S}

\section{Stack Frame}
A continuación se mostrarán los stack frames respectivos a cada función. Los casilleros en verde corresponden a los parámetros que serán pasados por la función caller, por lo que no son parte del stack frame de la función descripta en esa sección.
En los casos que aparezca un '-', se trata de padding y está solamente para mantener el tamaño de cada área como un múltiplo de '8'.
\subsection{Stack Frame byte encoder}
\renewcommand
{\tabcolsep}{12pt}
\begin{center}
\begin{tabular}{r | c |}
\cline{2-2}
20 &	   \cellcolor{green}a1 \\ \cline{2-2} 
16 &	   \cellcolor{green}a0 \\ \cline{2-2}
12 &  fp \\	\cline{2-2}
8  &  gp \\	\cline{2-2}
4  &  LH \\	\cline{2-2}
0  &  HL \\	\cline{2-2}
\cline{2-2}
\end{tabular}
\end{center}

\subsection{Stack Frame encode}
\renewcommand
{\tabcolsep}{12pt}
\begin{center}
\begin{tabular}{r | c |}
\cline{2-2}
52 &	  \cellcolor{green}a1 \\ \cline{2-2} 
48 &	  \cellcolor{green}a0 \\ \cline{2-2} 
44 &	  - \\ \cline{2-2} 
40 &	  ra \\ \cline{2-2}
36 &  fp \\	\cline{2-2}
32  & gp  \\	\cline{2-2}
28  &  s[1]\\	\cline{2-2}
24  &  s[0] \\	\cline{2-2}
20 &	 byte \\ \cline{2-2} 
16 &	 caracter \\ \cline{2-2}
12 &  a3 \\	\cline{2-2}
8  &  a2 \\	\cline{2-2}
4  &  a1 \\	\cline{2-2}
0  &  a0 \\	\cline{2-2}
\cline{2-2}
\end{tabular}
\end{center}

\subsection{Stack Frame correr Referencia}
\renewcommand
{\tabcolsep}{12pt}
\begin{center}
\begin{tabular}{r | c |}
\cline{2-2}
8  &  \cellcolor{green}a0 \\	\cline{2-2}
4  &  fp \\	\cline{2-2}
0  &  gp \\	\cline{2-2}
\cline{2-2}
\end{tabular}
\end{center}

\subsection{Stack Frame byte decoder}

\renewcommand
{\tabcolsep}{12pt}
\begin{center}
\begin{tabular}{r | c |}
\cline{2-2}
44 &	 \cellcolor{green}a1 \\ \cline{2-2} 
40 &	  \cellcolor{green}a0 \\ \cline{2-2}
36 &  - \\	\cline{2-2}
32  & ra \\	\cline{2-2}
28  &  fp \\	\cline{2-2}
24  &  gp \\	\cline{2-2}
20 &	  LN \\ \cline{2-2} 
16 &	  HL \\ \cline{2-2}
12 &  a3 \\	\cline{2-2}
8  &  a2 \\	\cline{2-2}
4  &  a1 \\	\cline{2-2}
0  &  a0 \\	\cline{2-2}
\cline{2-2}
\end{tabular}
\end{center}

\subsection{Stack Frame decode}
\renewcommand
{\tabcolsep}{12pt}
\begin{center}
\begin{tabular}{r | c |}
\cline{2-2}
52 &	  \cellcolor{green}a1 \\ \cline{2-2} 
48 &	  \cellcolor{green}a0 \\ \cline{2-2} 
44 &	  - \\ \cline{2-2} 
40 &	  ra \\ \cline{2-2}
36 &  fp \\	\cline{2-2}
32  & gp \\	\cline{2-2}
28  &  c \\	\cline{2-2}
24  &  caracter2 \\	\cline{2-2}
20 &	  byte \\ \cline{2-2} 
16 &	  caracter \\ \cline{2-2}
12 &  a3 \\	\cline{2-2}
8  &  a2 \\	\cline{2-2}
4  &  a1 \\	\cline{2-2}
0  &  a0 \\	\cline{2-2}
\cline{2-2}
\end{tabular}
\end{center}
\section{Conclusiones}
Realizando este trabajo práctico logramos terminar de afianzar los conocimientos adquiridos durante la cursada en lo que respecta a la programación en assembly.
Ya que al comienzo de este trabajo se poseía la implementación en C del mismo programa, se pudo compilar el mismo a assembly y compararlo con el código propio. Se comprobó de esta forma que los stack frames creados por el compilador reservaban mucho más espacio del necesario. A su vez, se pudo notar que programando directamente en assembly, el usuario tiene mayor flexibilidad para optimizar el software que en lenguajes de mayor nivel, a cuestas de la pérdida total de la independencia de la arquitectura a utilizar. 
\end{document}
